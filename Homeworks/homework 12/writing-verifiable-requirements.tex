\documentclass[a4paper,10pt]{article}
\usepackage[utf8]{inputenc}
\usepackage{dirtytalk}
\usepackage{mdwlist}
\usepackage{enumitem}
\def\arraystretch{1.2}%
\newenvironment{packed_enum}{
\begin{itemize}
  \setlength{\itemsep}{0pt}
  \setlength{\parskip}{0pt}  
  \setlength{\parsep}{0pt}
}{\end{itemize}}
\newenvironment{packed_dict}{
\begin{itemize}
  \setlength{\itemsep}{1pt}
  \setlength{\parskip}{0pt}  
  \setlength{\parsep}{0pt}
}{\end{itemize}}
\usepackage[a4paper,
	bindingoffset=0.2in,%
        left=1.2in,
        right=1.2in,
        top=1in,
        bottom=0.5in,%
        footskip=.25in]{geometry}
{
    \fontfamily{lmss}\selectfont 

    %opening
    \title{
        Writing verifiable requirements
    }

    \author{
        Jose Cruz
    }

    \date{
        Stevens Institute of Technology \\
        \small\today
    } 

    \begin{document}
    \maketitle
}

\section{Key Objectives}
The main objective of this application is to enable content creators to upload 
videos in a short format, no longer than 15 seconds, and to have viewers to be 
able watch, download, comment and like the content published by the creators.

\section{Requirements}
\subsection{User Requirements}
\begin{enumerate}[label=U-\arabic*]
    \item As a creator I want to publish videos.
    \item As a viewer I want to be able navigate between videos.
    \item As a viewer I want download videos from the application.
\end{enumerate}

\subsection{System Requirements}
\begin{enumerate}[label=SY-\arabic*]
    \item The system shall enable viewers and creators to comment on a video.
    \item If a video that is playing ends, the system shall repeat the video 
    from the beginning.
    \item If the creator enables it, the system shall allow a viewer to 
    download a video.
\end{enumerate}

\pagebreak

\subsection{Quality Requirements}
\subsubsection{Usability}
\begin{enumerate}[label=USE-\arabic*]
    \item A user shall be able to navigate between videos using gestures.
    \begin{enumerate}[label=USE-1.\arabic*]
        \item \textbf{Swipe Up}: To view the next video in the queue
        \item \textbf{Swipe Down}: To view the previous video in the queue, 
        if its the first video in the queue, the system shall empty the queue 
        and fetch new videos.
    \end{enumerate}
\end{enumerate}

\subsubsection{Portability}
\begin{enumerate}[label=POR-\arabic*]
    \item At least 70\% of the code base should be share between the Android
    and iOS source code. \textit{Programming Language: Kotlin}
\end{enumerate}

\subsubsection{Performance}
\begin{enumerate}[label=PER-\arabic*]
    \item The application should take no longer than 3 seconds to load the 
    main screen while on Wi-Fi and no longer than 5 seconds while on 
    cellphone data (4G).
\end{enumerate}

\pagebreak

\section{Requirement Feasibility}
\textbf{SY-3} - \say{If the creator enables it, the system shall allow a 
viewer to download a video} 

\subsection{Assumptions}
\begin{itemize}
    \item The video loaded successfully
    \item The application is caching on disk a video that was previously loaded.
    \item The device has a functional touch screen.
    \item The application is not frozen. \textit{Do not respond to user input}.
    \item The device have enough disk space to store the video.
    \item The device have enough available RAM to perform the download.
    \item The application have access to the device file system.
    \item The user allows the application `Write to disk` permission.
\end{itemize}

\subsection{Description}
If all the assumptions happens, the video is already in cache of the device, 
both iOS and Android have a sandbox area in the file system, so the application
do not have direct access to the main device file system, since the user 
already allows access to the application to write to the main file system, the 
only thing that we need to do is copy the file from the sandbox area to main
device file system. As long as the device have enough storage 
to save the video twice, once in the cache and the other one in the file 
system, the application should be able to export the video.

\pagebreak

\section{Requirement Verifiability}
\textbf{USE-1} - \say{A user shall be able to navigate between videos 
using gestures.}


\subsection{Test Case 1: Regular Flow}
The user touch the application icon from the home screen in their device, the 
application load the splash screen, the application start rendering the layout,
while it loads the video in the background, the video start playing, 
the user swipe up, the current video stop playing and disappears going up while 
the next video takes the place of the previous one, and start loading the video 
content. The user swipe down, the current video stop playing and disappear by 
moving to the bottom, the previous video comes from the top and starts playing.

\subsection{Test Case 2: Pull to refresh}
The user touch the application icon from the home screen in their device, the 
application load the splash screen, the application start rendering the layout,
while it loads the video in the background, the video starts playing, the user 
swipe down, the application displays a loading animation on top of the video, 
the video stop playing, a new video takes the place of the existing video and 
starts playing. In the background the application empty the previous video 
queue, and fetch a new one.

\end{document}