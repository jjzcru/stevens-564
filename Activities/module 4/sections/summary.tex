\section{Summary of Impacts}
\subsection{Operational impacts}

\subsubsection{Changes in procedure}
Customers can opt to use the self checkout system rather than go to a 
traditional checkout with a cashier and register, allowing them to pay for 
items at their own convenience as well as reduce lines for the 
checkout sections.

\subsubsection{Use of new data sources}
The new self checkout machines will have their own source of logging in items 
being bought and brought out of the store, allowing for another system of 
logging inventory other than manual count and the electronic cash registers

\subsubsection{Changes in quantity, type, and timing of data to be input 
into the system}
The rate at which objects are logged into the system for being bought from 
the store is affected by the varying paces of multiple customers making their 
own purchases through the self checkout system.


\subsubsection{Changes in data retention requirements}
Allowing for customers to scan their own items and input their cash or credit 
cards on their own allows the customers to input their own data directly 
into the stores computer systems.

\subsubsection{Changes in operational budget}
More funding will have to go towards paying for these self checkout machines 
to be installed as well as regular maintenance of the machines and 
repairs in case of breakdowns.

\subsubsection{Changes in operational risks}
Some customers will not understand the self checkout system or improperly use 
it costing the time of staff to help them throughout 
the checkout process. \newline

Issues with the self checkout machines breaking down and having to be 
repaired, costing both time and money.

\subsection{Organizational impacts}

\subsubsection{Modification of responsibilities}
More responsibilities will be focused on maintaining the functionality of 
the self checkout machines, basic personnel will have to understand the 
workings of the machines and can teach unwilling customers 
how to properly use it.

\subsubsection{Addition or elimination of job positions}
Adds the possible job position of Self Checkout Machine Specialists who 
regularly maintain and update the self checkout machines.

\subsubsection{Training or retraining users}
New employees will have to be trained in using and teaching the use of the 
self checkout system, older employees will also 
have to learn this as well. \newline

Customers who do not already understand the machine will have to be taught 
how to work the machine through store employees.

\subsubsection{Changes in numbers, skill levels, position identifiers, or 
locations of personnel}

New employees will have to fill positions for being able to regularly 
keep maintenance on the self checkout machines. \newline

These new employees will have to have separate, more qualified skillsets 
compared to cashiers or other store staff.

\subsection{Impacts during development}

\subsubsection{Parallel operation of the new and existing systems}
The self checkout machines will act similar to that of the already existing 
cash registers except the use of a cashier will not be required. \newline

Customers are expected to scan their own items, bag them accordingly, and 
input their money into the machine without requiring the direct involvement 
of the cashier.

\subsubsection{Operational impacts during system testing of the proposed system}
Any bugs or malfunctions in the system should be disposed of and fixed 
accordingly in preparation for the overall unveiling of the self 
checkout machines in stores for customer use.