\section{Validation }
At this point in time we have a list of all requirements that the client would 
like to see in the project. We do a review of the list and we start discarding 
the requirements that are not feasible, because of technical, geographical or 
monetary reasons. \newline

\noindent \textit{Since this project is a CRUD application \cite{crud} and do 
not require anything special from the hardware besides GPS and internet 
connection, most of the requirements are feasible, with the exception of 
some of performance requirements, were the client 
requested 15ms on page load.}\newline

\noindent \textit{Latency is an issue in distributed systems, making the 
requirements infeasible because of physics laws}. \newline

\noindent Next we ask them to put each system and requirement into the 
following categories:
\begin{itemize}
    \item \textbf{MVP} \cite{mvp}
    \item \textbf{Future Release}
\end{itemize}

\noindent For the \textbf{MVP} we ask them which features and systems are 
required for their release date, we ask them to be as objective as possible. 
Anything that is not in the \textbf{MVP} is going to 
\textbf{Future Release}. \newline

\noindent We focus the conversation over the MVP, now that we know which 
features and systems are important we need to prioritize them,  we send the 
client a document with all the features, grouped by system, and ask them to 
rank them between 1 and 5.. \textit{1 being the highest priority}. \newline

\noindent We use \textbf{dialog maps} to validate the workflow for each of 
the systems, the dialog maps are available in the Appendix. 
\begin{itemize}
    \item For the \textbf{Personal Shopper System} we used 
    the Figure \ref{fig:pss}. 
    \item For the \textbf{Personal Shopper Driver System} we used 
    the Figure \ref{fig:psds}. 
    \item For the \textbf{Personal Shopper Merchant System} we used 
    the Figure \ref{fig:psms}. 
\end{itemize}
