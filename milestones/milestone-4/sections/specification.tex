\section{Specification}
From the conversation that we have with our customers, they have particular 
keywords that they use that help determine what kind of requirements they 
are talking about. \newline

\noindent If they use terms like market share, revenue, sales or terms that 
are associated with money, we know that they are business requirements. \newline

\noindent If the sentences include a functionality or a feature or something 
that the system should be able to do is a \textbf{user requirement}, and during 
the meeting we try to decompose those into what are the inputs and the expected 
outputs of that functionality so we can use them later for validation and to 
write the system requirements properly. \newline

\noindent When we are talking about a particular feature, if the customer 
specifies how it should be a particular function we categorize those as 
\textbf{functional requirements}. \newline

\noindent In this project they have two main focuses, regarding quality, these are 
\textbf{speed} and \textbf{security}, anytime they mention the word fast, or 
private data, we know that they are talking about quality attributes. \newline

\noindent During the interviews we try to push them to figure out the system 
edge cases, and also we ask them to imagine how the application should work 
under certain conditions, this would be our \textbf{constraints}. \newline

\noindent To figure out how many systems are we building, we use the previous 
exercise to imagine how the application would work, the most frequent words 
they use to describe the system is browser and apps, so we ask for 
clarification on which browser, \textit{Mobile or Desktop}, and which 
platform, \textit{iOS or Android}, are they talking about, this ones would be ours 
\textbf{external interface requirements}. \newline

\noindent In general, during the meeting anytime one of the customers uses 
words like \textit{standard} or \textit{normal} we ask them to clarify those 
to avoid ambiguity. \newline

\noindent When they use terms like, \textit{perhaps}, \textit{maybe}, 
\textit{could}, \textit{sometimes}, we move the conversation in a way that 
they can evaluate all the possible cases and define what is the precise 
requirements.  \newline

\noindent If they are not able to articulate it, we take notes of that as a 
comments and we tell them that we can't add that into the system unless they 
define how the system should behave.

\pagebreak

\noindent If they provide some type of metric either distance or time, we ask 
them to specify the unit and the values, this are the
\textbf{performance requirements} \newline

\noindent If they offer a solution to a problem or they make a feature
request, we ask them:
\begin{packed_enum}
    \item How that solution or feature would help the user?
    \item How the implementation would work?
    \item What would be the required steps?
    \item How the system should managed each edge cases?
    \item Do we need to track new data for this feature?
    \item Does that feature affects existing requirements?
\end{packed_enum}
 
\noindent Sometimes the solution is too ambiguous and the customers decide that they do 
not understand the solution well enough to be added it to the requirements, 
other times that solution creates new requirements and also changes 
existing ones. \newline
