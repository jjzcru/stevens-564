\section{Analysis}
After each meeting we review the notes, try to reply the following questions.
\begin{packed_enum}
    \item Are there any new requirements?
    \item The new requirements conflicts with existing one?
    \item What kind of requirements are this new requirements?
    \item Do we need to update any existing requirements?
    \item Do we know all the possible edge cases?
    \item Do we know how to solve the edge cases?
\end{packed_enum}

\subsection{Mission Statement, Key Drivers, Key Constraints}
The goal for the meeting was to determine:
\begin{packed_enum}
    \item What is their motivation to create this project?
    \item What problem in the market do they want to fulfill?
    \item How do they want to achieve this?
    \item What is the business model?
    \item How big is the market for this project, short and long term?
    \item Who are their competitors?
    \item What is their product differentiator?
    \item What kind of constraints do they have, either time, 
    technological or monetary?
\end{packed_enum}

\noindent We had a meeting in a free form setting, brainstorm, so the customers could 
provide their insight without any specific agenda, sometimes one or more of the 
people in the group focused on an area in particular and we moved the 
conversation over that topic.

\pagebreak
\subsection{Key Stakeholders, User Requirements, Business Requirements}
Once we established why the client would like to develop this project we went 
ahead and tried to figure out who are the stakeholders, what is the scope of 
the project and what business goals are they trying to achieve. \newline

\noindent The goal for the meeting was to determine:
\begin{packed_enum}
    \item How many stakeholders are in the project?
    \item What is the stakeholder main motivator?
    \item How the project would help the stakeholders?
    \item What are their business goals in the short, medium and long term?
    \item How many systems do we need to build for this project?
    \item What functionalities do each stakeholder want? 
    \item Seize what is the scale on which the program is going to run in the 
    short, medium and long term
    \item What is their budget?
    \item What is their go to market timeline? \textit{To stablish a deadline.}
\end{packed_enum}

\noindent We sent the clients a document with the questions that we had from 
the previous meeting, some were to make sure that we understood their 
motivation, others were to get clarification because some of the replies 
seemed ambiguous.\newline

\noindent We used an interview approach to get these clarifications, this 
process was quick so we moved the conversation towards getting to know what is 
their current status regarding resources and constraints, what is their budget, 
what is their business goals, how much traffic they were expecting.\newline

\noindent We use this information to create their business requirements. Once 
we have these we start talking about the key stakeholders for the program. 
\begin{packed_enum}
    \item What was their motivation? 
    \item What is their target audience? \textit{Demographic}
    \item What is their social-economic status? 
    \item How are they solving their problem right now, and how the project 
    would improve the existing solution.
\end{packed_enum}


\noindent We move from a interview format towards brainstorming, we want to open 
the scope as much as possible to have place for creativity and during 
the conversation we start closing the scope by getting into more 
details.\newline

\noindent We achieve this by discarting ideas that were not feasible, 
either from a technological perspective or time constraints.

\pagebreak

\noindent With this we identify the key stakeholders, and then we start 
discussing what kind of functionalities each of these stakeholders would want 
to achieve with the program, this was still in a brainstorming format. \newline

\noindent We used this information to generate the User Requirements.

\subsection{System Requirements, Security Requirements, Quality Requirements}
With the information from the previous meeting some problem arises, there was 
conflict in some of their requirements, we take notices of those and we put 
them in a document that we sent before the meeting. \newline

\noindent The goal for the meeting was:
\begin{packed_enum}
    \item Clarification of the conflicting requirements
    \item Use case validation
    \item Quality attributes
    \begin{itemize}
        \item Performance
        \item Accessibility 
        \item Usability
        \item Security
    \end{itemize}
    \item How many platforms are going to be develop?
    \item What kind of constraints do they have regarding providers?
    \item What kind of compliance the software requires?
\end{packed_enum}

\noindent From the previous meeting we know what is the goal and motivation 
for the stakeholders, and also how they interact with one another, now the 
focus is moving towards how can we achieve those expectations 
using software. \newline

\noindent We as a team already have an idea on how the project could be 
implemented and which providers to use to achieve it. \newline

\noindent We were giving suggestions on how the system could be implemented. \newline

\noindent Now it is time to transform these user requirements into system 
requirements, we start asking them questions about what the process should be 
and ask them on how the system should behave in case the system reaches a 
particular edge case. \newline

\noindent During the conversation we ask questions on how to validate if this use case 
was executed successfully, these were the foundations for the 
quality requirements. \newline

\pagebreak

\noindent Once we understood how can we verify if the requirements were 
successful or not, we move the conversations towards how good this were 
supposed to be, we start asking questions about:
\begin{itemize}
    \item How fast X should be able to perform
    \item How many steps should happen to achieve a particular use case?
    \item How should we do data validation?
    \item What is the expected uptime for a particular system?
    \item When are the expected peak hours, if any?
\end{itemize}

\noindent We take these notes and create a formal representation of these requirements 
using the grammar used in the textbook.




